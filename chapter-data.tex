% !TEX root = ../my-thesis.tex
%

\chapter{Data Collection}
\label{sec:data}
For this paper, coronavirus cases for two countries, Germany and Norway, are analysed. The data for Norway comes from two official sources, the Institute of Public Health and the Norwegian Directorate of Health\autocite[Cf.][]{fhi}. For each Norwegian municipality, the daily number of infections is published, dating back to 26 March 2020. For Germany, the Robert Koch Institute\autocite[Cf.][]{rki} provides data starting on 2 January 2020. These two data sources are updated daily. The shapefiles for the Norwegian and German municipalities come from Geonorge\autocite[Cf.][]{geonorge} and Esri Germany\autocite[Cf.][]{opendata}, respectively. Both Apple and Google\autocite[Cf.][]{google_mobility} provide daily mobility reports for different countries around the world. All mobility reports are broken down by location and show how the number of visits to places like grocery shops and parks has changed. For Norway this data is available at the municipality level and for Germany at the state level. For this work, only Google's mobility reports were used. Finally, OpenStreetMap\autocite[Cf.][]{OpenStreetMap} is a free project that collects, structures and stores freely usable geodata in a database for use by anyone. By querying this database, lists of various points of interest can be collected, e.g. the number of churches in the city of Munich. To do this, a bounding box is drawn around each municipality, the data for this area is downloaded using the R package \textit{osmdata}\autocite[Cf.][]{osmdata} and then the points are spatially matched to the original shape again. \\
Below is an example of how the data for the Trondheim municipality was collected.
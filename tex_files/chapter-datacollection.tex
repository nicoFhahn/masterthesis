% !TEX root = ../my-thesis.tex
%
\chapter{Dataset Collection}
As is often the case with statisticians, the construction of the dataset used to analyse a research question is an essential task and frequently involves the merging of multiple data sources to create a dataset. This was the case in this thesis and in the following chapter a brief overview of the data sources used, their pre-processing and how they were combined is given. All of the data that is used in this work is openly available, either through government sources or Github.
\label{sec:datacollection}
\clearpage
\section{Covid-19 Data}
\subsection{Covid-19 Data for Norway}
The Covid-19 data for Norway comes from a dataset made available to the public via the a repository on the website \href{https://www.github.com}{Github.com}, created by the user thohan88. The repository contains a daily updated dataset that is the result of combining several data sources, which include the Institute of Public Health and the Norwegian Directorate of Health. According to the author of the repository, the project is "an open-source effort to make data about the Covid-19 situation in Norway available to the public in a timely and coherent manner" \autocite[][]{thohan88}. \\
A few sample data points from this dataset are displayed in Table~\ref{datasetNorge}.\\
\begin{table}[H] 
\caption{An excerpt from the Covid-19 data for Norway. Does not contain all variables. The number of infections are the cumulative number of infections. \label{datasetNorge}}
\begin{tabular}{l l r r r}
\toprule
\textbf{Id}	& \textbf{Municipality}	& \textbf{Population}	& \textbf{Inf. 2020-03-26}	& \textbf{Inf. 2020-03-27}\\
\midrule
1103 & Stavanger & 143574 & 87 & 88 \\
1507 & Ålesund & 66258 & 20 & 20 \\
4601 & Bergen & 283929 & 231 & 248 \\
5001 & Trondheim & 205163 & 113 & 136 \\
\bottomrule
\end{tabular}
\end{table}
\subsection{Covid-19 Data for Germany}
In Germany, the Robert Koch Institute publishes daily situation reports in which the number of new cases is published at NUTS 3 level. These reports are available as pdf files via the Institute's website. They can be downloaded and grouped via the R package \texttt{covid19germany}\autocite[][]{covid19germany}, as was done for this work.\\
A few sample data points from this dataset are displayed in Table~\ref{datasetGermany}.\\
The variable \textit{CumNumberTestedIll} contains the cumulative number of people that have tested positive for Covid-19.
\begin{table}[H] 
\caption{An excerpt from the Covid-19 data for Germany. Does not contain all variables.\label{datasetGermany}}
\begin{tabular}{l l r r r}
\toprule
\textbf{Municipality}	& \textbf{Date}	& \textbf{Cumulative number of infections} & \textbf{Population}\\
\midrule
SK München & 2020-01-29 & 1 & 1471508\\
SK München & 2020-02-03 & 2 & 1471508\\
SK München & 2020-02-11 & 3 & 1471508\\
LK Rosenheim & 2020-02-29 & 1 & 260983\\
LK Rosenheim & 2020-03-08 & 2 & 260983 \\
LK Rosenheim & 2020-03-10 & 6 & 260983 \\
\bottomrule
\end{tabular}
\end{table}
\clearpage
\section{Vaccination Data}
At the end of 2020, the first countries began vaccinating against Covid-19. Vaccination leads to a milder course of the disease, but it is not yet known whether vaccinated people can continue to transmit Covid-19, as of early May 2021. By including the proportion of people vaccinated, one can give an indication of the extent to which vaccination helps prevent the spread of Covid-19. These data are available for Norway at the municipality level via the statistics database of the National Institute of Health, FHI \autocite[][]{fhi}. For Germany, these data are only available at the state level and are therefore not used.
\clearpage
\section{Demographic Data}
As demographics tend to differ between different geographic units, the decision was made to include demographic variables in the analysis of the research question to see if the risk for infection may be higher when a certain characteristic is present in the population.
\subsection{Demographic Data for Norway}
The demographic data collected for Norway comes from Statistisk Sentralbyrå and is made available to the public through their online database, StatBank\autocite[][]{ssb}. \\
The first characteristic collected was the age of the population in a given municipality. For each age, starting at 0 and ending at 105, the number of people of that age is known. \\
Next, unemployment data were collected for a given municipality. For each municipality, the percentage of all people out of work is known, as well as the percentage of all immigrants out of work. \\
Other data collected include data related to the number of workers in a particular industry, as well as immigration data. Since there is discussion about whether workers from certain industries, in this case construction, contribute to the spread of Covid-19, the decision was made to collect this type of data. For each community, the number of workers across all industries is known, as well as the number of workers in the construction industry. Workers, in this case, are individuals employed in a given municipality who are between the ages of 20 and 66. It is also known how many people work full-time and how many work part-time. \\
Finally, for immigration data, it is known how many immigrants live in a given municipality and how many Norwegians were born to immigrant parents. These figures are known in terms of the percentage of the population in 2020.
\subsection{Demographic Data for Germany}
The demographic data collected for Germany comes from the federal and state statistical offices and is made available to the public through their online database, Regionaldatenbank Deutschland \autocite[][]{rdb}. \\
The first characteristic collected was unemployment data at the NUTS 3 level. For each municipality, the number of unemployed people as well as the number of unemployed foreigners was collected. \\
Next, data related to the European elections in 2019 were collected. In each municipality, it is known how many people voted in total, how many people voted for the six largest parties, and how many votes the remaining parties received combined. \\
Data was also collected in relation to people seeking protection, welfare recipients and in relation to asylum seeker benefits. It is known how many people sought protection in Germany, how many received social welfare and how many received asylum seeker benefits. \\
Finally, trade tax, income tax, and payroll tax data were collected for each municipality. 
\clearpage
\section{Shapefiles}
In addition to numeric variables, the dataset also contains a geographic variable containing the geographic boundaries of a given municipality or city/district.
\subsection{Shapefiles for Norway}
The data for the Norwegian shapefiles comes from Geonorge \autocite[][]{geonorge} and is downloaded from a Github repository, as the data there was in a cleaner state \autocite[][]{shapeGithub}. In addition to the geographic shape, the dataset also includes a variable that contains the ID of each municipality.
\subsection{Shapefiles for Germany}
The data for the German shapefiles comes from Esri Germany \autocite[][]{esri}.
\clearpage
\section{OpenStreetMap Data}
OpenStreetMap (OSM) is a free project that collects, structures and stores freely usable geodata in a database for use by anyone (Open Data). This data is available under a free license, the Open Database License. The core of the project is therefore an openly accessible database of all contributed geoinformation \autocite[][]{OpenStreetMap}. \\
In R, the OpenStreetMap API can be queried using the R package \texttt{osmdata} \autocite[][]{osmdata}. To download all locations of a given type in a given region, a shape or bounding box must be specified along with a key and optionally a value. These key-value pairs are used to specify the type of location, for example, the "amenity" key is used for all facilities used by visitors and residents. If you use the "biergarten" value together with the "amenity" key, the locations of all beer gardens in a given geographic region will be downloaded. \\
OpenStreetMap users have the option to map a location as either \texttt{POINT}, \texttt{POLYGON}, \texttt{MULTIPOLYGON}, \texttt{LINESTRING}, or \texttt{MULTILINESTRING}. Conventionally, the first three are used. Therefore, only sites mapped as one of these were used for this work. If a location was mapped as either \texttt{POLYGON} or \texttt{MULTIPOLYGON}, the centroid of the location was calculated. \\
A complete list of all key-value pairs used for this work can be found in the Appendix. 
\clearpage
\section{Data Wrangling}
The final step before analysing the research question at hand is to combine all of these data sources into one dataset. This section will show how this was achieved.
\subsection{Data Wrangling for Norway}
The initial step in creating the final dataset was to convert the data from a wide format, as seen in Table~\ref{datasetNorge}, to a long format. This was done using the function \texttt{melt()} from the R package \texttt{reshape2} \autocite[][]{reshape2}. The long version of the dataset is shown in Table~\ref{norwayLong}.
\begin{table}[H] 
\caption{An excerpt from the long version of the Norwegian Covid-19 data. Does not contain all variables.\label{norwayLong}}
\begin{tabular}{l l r r r}
\toprule
\textbf{Id}	& \textbf{Municipality}	& \textbf{Population} & \textbf{Date} & \textbf{Infections}\\
\midrule
1507 & Ålesund & 66258 & 2020-03-26 & 20\\
5001 & Trondheim  & 205163  & 2020-03-26 & 113\\
1507 & Ålesund & 66258 & 2020-03-27 & 20\\
5001 & Trondheim  & 205163  & 2020-03-27 & 136\\
\bottomrule
\end{tabular}
\end{table}
Next, the demographic data for Norway was loaded and processed. Since the age data contains the number of people of a certain age, the median age was calculated for each region based on how many people of each age group live in each region. \\
The other demographic variables are left unchanged. To combine the demographic data with the Covid-19 data, the municipality IDs were extracted using the \texttt{str\_extract()} function from the \texttt{stringr} \autocite[][]{stringr} R package using the regular expression \texttt{[0-9]\{4\}}. Next, all demographic datasets and the Covid-19 dataset were merged using the \texttt{merge()} function. \\
Using the \texttt{st\_intersects()} function from the \texttt{sf} \autocite[][]{sf} R package, the number of points of interest downloaded via OpenStreetMap was calculated for each municipality. Since the shapefiles contain the ID for each community, these data were then merged with the data containing the demographic and Covid-19 data. \\
For each numeric variable, e.g. the number of schools or the number of employees, this number was scaled. \\
If there were missing values in the covariates, these values were imputed using the median of the respective variable. \\
Next, the vaccination data for Norway was loaded. As the daily number of vaccinated persons for each municipality is included in the data, these numbers only needed to be cumulated before being merged with the rest of the data based on municipality name and date. \\
Finally, seven new variables were created:
\begin{itemize}
    \item[1.] Expected count, which is the expected number of cases in each municipality.
    \item[2.] SIR, which is the standardised incidence ratio in each municipality.
    \item[3.] An area ID, which is a unique ID given to each municipality.
    \item[4.] Higher education, which counts the number of universities and colleges in a given area.
    \item[5.] Sex, which gives the proportion of females living in a given area.
    \item[6.] Population density, i.e. the number of people per square kilometre in a given area.
    \item[7.] Urban density, i.e. the number of residential buildings per square kilometre in a given area.
\end{itemize}
The final dataset contains the variables shown in Table~\ref{datasetNorway}.
\begin{table}[H] 
\caption{The variables contained in the final dataset.\label{datasetNorway}}
\begin{tabular}{l l l}
\toprule
\textbf{Variable Name}	& \textbf{Explanation}	& \textbf{Scale}\\
\midrule
Id & The municipality ID & None \\
Municipality & The municipality name & None \\
Population & Population in a municipality & None \\
Date & The date of the data used & None \\
Infections & The number of infected people & None \\
Median age & The median age & scaled \\
\multirow{2}{*}{Total unemployment} & The proportion of &\multirow{2}{*}{scaled}\\
& unemployed people \\
\multirow{2}{*}{Unemployed immigrants} & The proportion of & \multirow{2}{*}{scaled}\\
 & unemployed immigrants  \\
\multirow{2}{*}{Full-time workers} & The number of & \multirow{2}{*}{scaled} \\
& full-time workers \\
\multirow{2}{*}{Part-time workers} & The number of & \multirow{2}{*}{scaled} \\
& part-time workers \\
\multirow{2}{*}{Full-time construction} & The number of full-time & \multirow{2}{*}{scaled} \\
& construction workers \\
\multirow{2}{*}{Part-time construction} & The number of part-time & \multirow{2}{*}{scaled} \\
& construction workers \\
Total immigrants & The proportion of immigrants  & scaled \\
Marketplace & The number of marketplaces & scaled \\
Entertainment & The number of entertainment venues & scaled \\
Sport & The number of sports amenities & scaled \\
Clinic & The number of clinics & scaled \\
Hairdresser & The number of hairdresser & scaled \\
Shops & The number of shops & scaled \\
Place of worship & The number of places of worship & scaled \\
Retail & The number of retail stores & scaled \\
Nursing home & The number of nursing homes & scaled \\
Restaurant & The number of restaurants & scaled \\
Aerodrome & The number of aerodromes & scaled \\
Office & The number of offices & scaled \\
\multirow{2}{*}{Platform} & The number of public & \multirow{2}{*}{scaled} \\
& transport platforms \\
Kindergarten & The number of kindergartens & scaled \\
Schools & The number of schools & scaled \\
Bakeries & The number of bakeries & scaled \\
Residential & The number of residential buildings & None \\
\multirow{2}{*}{Higher education} & The number of colleges & \multirow{2}{*}{scaled} \\
& and universities \\
Expected count & The expected number of infections & None \\
SIR & The standardised incidence ratio & None \\
Area Id & A unique ID & None \\
Area & The area in km$^2$ & None \\
Population density & People per km$^2$ & scaled\\
\multirow{1}{*}{Urban density} &  Residential buildings per km$^2$ & scaled \\
Sex & The proportion of females & scaled \\
Vaccinations & The proportion of vaccinated & scaled \\
\bottomrule
\end{tabular}
\end{table}
\subsection{Data Wrangling for Germany}
The data processing procedure for Germany is identical to that for Norway. First, all demographic variables were loaded and left unchanged before being merged with the Covid-19 data prior to calculating the spatial intersections between the points of interest and the NUTS-3 areas. After merging all the data, the scaled numbers were calculated for the numeric variables. For the variables containing the number of people who voted for a particular political party, the relative percentage of votes the party received was calculated. Again, missing values were imputed using the median.
Finally, the same seven new variables were created. 
The final dataset contains the variables shown in Table~\ref{finalGermany}.
\begin{table}[H] 
\caption{The variables contained in the final dataset.\label{finalGermany}}
\begin{tabular}{l l l}
\toprule
\textbf{Variable Name}	& \textbf{Explanation}	& \textbf{Scale}\\
\midrule
Id & The municipality ID & None\\
Municipality & The municipality name & None \\
Population & Population in a municipality & None \\
Date & The date of the data used & None\\
Infections & The number of infected people & None \\
Logarithmic trade tax & The trade tax in Euros & scaled \\
Logarithmic income tax & The income tax in Euros & scaled \\
Logarithmic total income & The income and payroll tax in Euros & scaled \\
\multirow{2}{*}{Asyl benefits} & The number of people & \multirow{2}{*}{scaled} \\
& receiving asylum seeker benefits \\
Welfare recipients & The number of welfare recipients & scaled \\
Unemployed people & The number of unemployed people & scaled \\
Unemployed foreigners & The number of unemployed foreigners & scaled \\
Protection seekers & The number of protection seekers & scaled \\
Die Union & Percentage of vote for Union & scaled\\
SPD & Percentage of vote for SPD & scaled\\
Greens & Percentage of vote for the Greens & scaled\\
FDP & Percentage of vote for FDP & scaled\\
The left & Percentage of vote for the left & scaled\\
AfD & Percentage of vote voted for AfD & scaled \\
Marketplace & The number of marketplaces & scaled \\
Entertainment & The number of entertainment venues & scaled \\
Sport & The number of sports amenities & scaled \\
Clinic & The number of clinics & scaled \\
Hairdresser & The number of hairdresser & scaled \\
Shops & The number of shops & scaled \\
Place of worship & The number of places of worship & scaled \\
Retail & The number of retail stores & scaled \\
Nursing home & The number of nursing homes & scaled \\
Restaurant & The number of restaurants & scaled \\
Aerodrome & The number of aerodromes & scaled \\
Office & The number of offices & scaled \\
\multirow{2}{*}{Platform} & The number of public & \multirow{2}{*}{scaled} \\
& transport platforms \\
Kindergarten & The number of kindergartens & scaled \\
Schools & The number of schools & scaled \\
Bakeries & The number of bakeries & scaled \\
Residential & The number of residential buildings & None \\
\multirow{2}{*}{Higher education} & The number of colleges & \multirow{2}{*}{scaled} \\
& and universities \\
Expected count & The expected number of infections & None \\
SIR & The standardised incidence ratio & None \\
Area Id & A unique ID & None \\
Area & The area in km$^2$ & None \\
Population density & People per km$^2$ & scaled  \\
\multirow{1}{*}{Urban density} & Residential buildings per km$^2$  & scaled\\
Sex & The proportion of females & scaled \\
\bottomrule
\end{tabular}
\end{table}
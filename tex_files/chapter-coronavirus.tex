% !TEX root = ../my-thesis.tex
%
\chapter{Corona Virus}
\label{sec:corona}
Viral diseases continue to pose a serious public health threat. Several viral epidemics have occurred in the last 20 years, including the SARS pandemic in 2002/3, H1N1 influenza in 2009, and more recently the Middle East Respiratory Syndrome Coronavirus (MERS-CoV), which was first detected in Saudi Arabia in 2012. \\
In late 2019, the first few cases of lower respiratory infections were detected in Wuhan, China. In February 2020, this viral disease was officially named "Covid-19", an acronym for "Coronavirus Disease 2019". \\
Due to the rapid spread of the virus, a Public Health Emergency of International Concern was declared at the end of January 2020, with 18 countries reporting cases and four countries reporting human-to-human transmission. \\
At the end of February 2020, the World Health Organisation (WHO) raised the risk of a Covid 19 epidemic to "very high" before declaring it a pandemic on 11 March. At that time, more than 118,000 cases in 114 countries and 4000 deaths had already been registered. \\
The first cases of the disease were linked to direct exposure at the Huanan Seafood Wholesale Market in Wuhan, with animal-to-human transmission suspected as the main mechanism. After subsequent cases could not be linked to this mechanism, human-to-human transmission was presumed to be the main transmission mechanism. Furthermore, sympotomatic individuals are thought to be the most common source of covid-19 spread. However, asymptomatic individuals can also transmit the virus, therefore isolation is the best way to contain this epidemic. \\
Similar to other respiratory diseases, e.g. influenza, transmission is thought to occur through respiratory droplets (particles $>5-10\mu m$ in diameter) when coughing and sneezing. In closed rooms, transmission by aerosol is also possible. \\
Based on the data from the first cases in Wuhan, the incubation period is generally between 3 and 7 days, with a median of 5.1 days. According to the data, the number of infections doubled about every seven days and the basic reproductive number $R$ is 2.2, which means that on average each infected individual infects another 2.2 individuals. \\
According to a report by the Chinese Centre for Disease Control, which studied 72,314 cases, the overall mortality rate of confirmed cases was 2.3\%, with most of the fatal cases affecting people over 70 years of age. \\
Furthermore, the clinical manifestations of the disease can be divided into three groups according to their severity:
\begin{itemize}
    \item Mild disease: non-pneumonia and mild pneumonia; this occurred in 81\% of cases.
    \item Severe disease: dyspnea, respiratory rate $\geq 30$ min, blood oxygen level $\leq 93\%$; this occurred in 14\% of cases.
    \item Critical disease: respiratory failure, septic shock and/or multiple organ dysfunction or failure; this occurred in 5\% of cases.
\end{itemize}
Subsequent reports indicate that the disease is asymptomatic or with very mild symptoms in 70\% of patients, while the remaining 30\% develop a respiratory syndrome with high fever, cough and even severe respiratory failure, which may require admission to the intensive care unit. \\
Most countries use some kind of clinical and epidemiological information to determine who should be tested. A molecular test, for example a PCR test, can be used to detect the disease.\\
The WHO recommends the collection of samples from both the upper and lower respiratory tract. In the laboratory, the genetic material extracted from the saliva or mucus sample is amplified by reverse polymerase chain reaction (RT-PCR), which synthesises a double-stranded DNA molecule from an RNA form. Once the genetic material is sufficient, the parts of the genetic code of the CoV that are conserved are searched for. The probes used are based on the original gene sequence published by the Shanghai Public Health Clinical Center \& School of Public Health, Fudan University, Shanghai, China on Virological.org and subsequent confirmatory evaluation by other laboratories \autocite[][]{cascella2021features}.
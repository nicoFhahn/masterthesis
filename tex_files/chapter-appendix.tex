% !TEX root = ../my-thesis.tex
%
\chapter{Appendix}
\label{sec:appendix}

\section{The Negative Binomial Distribution}
The negative binomial distribution is a univariate probability distribution that belongs to the discrete probability distributions. It models the number of trials required to achieve a given number of successes in a Bernoulli process. \\
The density is given by
\begin{equation}
    f\left(k,r,p\right)=\mathcal{P}\left(X=k\right)=\begin{pmatrix} k+r-1\\r-1\end{pmatrix}\left(1-p\right)^kp^r,
\end{equation}
with $r$ the number of successes, $k$ the number of failures, and $p$ the probability of success \cite{haldane1941fitting}.

\section{Code examples}
\subsubsection{Specifying the Different Types of Models}
\begin{lstlisting}[caption={Specifying different models in INLA.}, label={codeModels}, language=R]
# Specify a penalized prior
prior_1 <- list(
  prec = list(
    prior = "pc.prec",
    param = c(1, 0.01)
  )
)
# Create a neighbourhood list
nb <- poly2nb(newest_numbers_germany)
# save the neighbourhood
nb2INLA("maps/map_2.adj", nb)
# load it
g <- inla.read.graph(filename = "maps/map_2.adj")
# specify the model formula for the bym2 model
formula_bym2 <- CumNumberTestedIll ~
  # add the demographic vars and pop density
  pop_dens + urb_dens + sex + 
  # specify the model with neighbourhood matrix
  f(
    idarea_1, model = "bym2", graph = g,
    scale.model = TRUE, hyper = prior_1
  )
# compute the model
res_bym2 <- inla(
  formula_bym2,
  family = "nbinomial",
  data = newest_numbers,
  E = expected_count,
  control.predictor = list(
    compute = TRUE
  ),
  control.compute = list(dic = TRUE, waic = TRUE, cpo = TRUE)
)
# specify the model formula for besags proper spatial model
formula_bp <- CumNumberTestedIll ~
  # add the demographic vars and pop density
  pop_dens + urb_dens + sex + 
  # specify the model with neighbourhood matrix
  f(
    idarea_1, model = "besagproper",
    graph = g, hyper = prior_1
  )
# compute the model
res_bp <- inla(
  formula_bp,
  family = "nbinomial",
  data = newest_numbers,
  E = expected_count,
  control.predictor = list(
    compute = TRUE
  ),
  control.compute = list(dic = TRUE, waic = TRUE, cpo = TRUE)
)
# compute the Q matrix
Q <- Diagonal(x = sapply(nb, length))
for(i in 2:nrow(newest_numbers)) {
  Q[i - 1, i] <- -1
  Q[i, i - 1] <- -1
}
# compute the C matrix
C <- Diagonal(x = 1, n = nrow(newest_numbers)) - Q
# specify the model formula for the leroux model
formula_leroux <- CumNumberTestedIll ~
  # add the demographic vars and pop density
  pop_dens + urb_dens + sex + 
  # specify the model with neighbourhood matrix
  f(
    idarea_1, model = "generic1",
    Cmatrix = C, hyper = prior_1
  )
# compute the model
res_leroux <- inla(
  formula_leroux,
  family = "nbinomial",
  data = newest_numbers,
  E = expected_count,
  control.predictor = list(
    compute = TRUE
  ),
  control.compute = list(dic = TRUE, waic = TRUE, cpo = TRUE)
)
\end{lstlisting}
\subsubsection{Variable Selection using INLA}
\begin{lstlisting}[caption={The code for variable selection in INLA.}, label={codeSelection}, language=R]
# define the stack
stack_all <- inla.stack(
  data = list(
  CumNumberTestedIll = newest_numbers$CumNumberTestedIll
    ),
  A = list(1),
  effects = list(
    data.frame(
      Intercept = 1,
      newest_numbers[, c(2:5, 8:18, 26:36, 38:40, 43:48, 56:59)]
    )
  )
)
# run the selection
result_backwards <- INLAstep(
  fam1 = "nbinomial",
  newest_numbers,
  in_stack = stack_all,
  invariant = "Intercept",
  direction = "backwards",
  include = c(2:5, 8:18, 26:36, 38:40, 43:48, 56:59),
  y = "CumNumberTestedIll",
  y2 = "CumNumberTestedIll",
  powerl = 1,
  inter = 1,
  thresh = 2
)
# run the selection
result_forwards <- INLAstep(
  fam1 = "nbinomial",
  newest_numbers,
  in_stack = stack_all,
  invariant = "Intercept",
  direction = "forwards",
  include = c(2:5, 8:18, 26:36, 38:40, 43:48, 56:59),
  y = "CumNumberTestedIll",
  y2 = "CumNumberTestedIll",
  powerl = 1,
  inter = 1,
  thresh = 2
)
\end{lstlisting}
\subsubsection{Infrastructure Models for Germany}
\begin{lstlisting}[caption={The code for the demographic models.}, label={infraGermanyCode}, language=R]
prior_2 <- list(
  prec = list(
    prior = "pc.prec",
    param = c(0.5 / 0.31, 0.01)
  )
)
formula_22_bym2 <- CumNumberTestedIll ~
  pop_dens + urb_dens + marketplace + entertainment + sport +
  clinic + toilet + hairdresser + shops + place_of_worship +
  retail + nursing_home + restaurant + aerodrome + office +
  platform + schools + higher_education + banks + kindergarten +
  bakeries + gas + atm +
  f(
    idarea_1, model = "bym2",
    graph = g, scale.model = TRUE,
    hyper = prior_2
  ) 
formula_23_bym2 <- CumNumberTestedIll ~
  marketplace + entertainment + sport + clinic +
  toilet + hairdresser + shops + place_of_worship + retail +
  nursing_home + restaurant + aerodrome + office + platform +
  schools + higher_education + banks + kindergarten + bakeries +
  gas + atm +
  f(
    idarea_1, model = "bym2", graph = g,
    scale.model = TRUE, hyper = prior_1
  ) 
formula_26_leroux <- CumNumberTestedIll ~
  marketplace + entertainment + clinic + toilet + hairdresser +
  place_of_worship + retail + nursing_home + restaurant +
  terminal + platform + kindergarten + schools + bakeries +
  gas + banks + atm + pop_dens + higher_education +
  f(
    idarea_1, model = "generic1",
    Cmatrix = C, hyper = prior_2
  )
\end{lstlisting}
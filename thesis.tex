\documentclass[12pt]{book}
\usepackage{amsmath}
\usepackage{amssymb}
\usepackage{hyperref}
\usepackage[backend=biber,style=alphabetic,citestyle=authoryear,autocite=footnote,citereset=section, maxcitenames=2]{biblatex}
\bibliography{references.bib}

\begin{document}
\chapter{Introduction}
\chapter{Corona Virus}
\chapter{The Modelling of Geospatial Health Data}
\section{Exponential Families and the Conjugate Prior}
In statistics and probability theory, and exponential family is a parametric set of probability distributions of a specific form. The distribution of a random variable $y$ is an exponential family, if the discrete or continuous density with respect to a $\sigma$-finite measure of $y$ follows the form
\begin{equation}
    f(y|\theta, \lambda)=\exp\left(\frac{y'\theta - b(\theta)}{\lambda}+c(y,\lambda) \right),
\end{equation}
with $c(y,\lambda)\geq 0$ and measurable. $\theta\in\Theta\subset\mathbb{R}^q$ is the \textit{natural} or \textit{canonical} parameter of the exponential family while $\lambda > 0$ is a \textit{dispersion} or \textit{nuisance} parameter. The natural parameter space $\Theta$ is the set of all $\theta$ satisfying $0<\int\exp\left(\frac{y'\theta - b(\theta)}{\lambda}+c(y,\lambda) \right)dy< \infty$. Furthermore, $b(\theta)$ is a twice differentiable function and all moments of $y$ exist. Specifically, 
\begin{alignat}{3}
    \mathbb{E}_\theta(y) &= \mu(\theta) =& \frac{\partial b(\theta)}{\partial\theta} \\
    \hbox{Cov}_\theta(y) &= \Sigma(\theta) =& \lambda\frac{\partial^2b(\theta)}{\partial\theta\partial\theta'}.
\end{alignat}
The covariance matrix $\Sigma(\theta)$ is positive definite in $\Theta^0$, therefore $\mu:\Theta^0\rightarrow  M = \mu\left(\Theta^0\right)$ is injective. By inserting the inverse function $\theta(\mu)$ into $\frac{\partial^2b(\theta)}{\partial\theta\partial\theta'}$, the variance function 
\begin{equation}
    v(\mu)=\frac{\partial^2b(\theta(\mu))}{\partial\theta\partial\theta'}
\end{equation}
is given and the covariance can be written as
\begin{equation}
    \hbox{Cov}_\theta(y) = \lambda v(\mu).
\end{equation}
Important exponential families are the normal, binomial, Poisson, gamma and inverse Gaussian distribution. \autocite[Cf.][]{fahrmeir2013multivariate} One property of exponential families is, that they have conjugate priors which is and important property in Bayesian statistics. HIERNOCHWASCONJUGATE PRIOR
\section{Latent Gaussian Models}
\section{Markov Chain Monte Carlo Methods}
\section{Integrated Nested Laplace Approximation}
\chapter{Data Collection}
\chapter{Data Analysis}
\chapter{Results}
\chapter{Final Thoughts}
\end{document}